\documentclass[a4paper,10pt]{scrartcl}

\usepackage[utf-8]{inputenc}
\usepackage[ngerman]{babel}
\usepackage[T1]{fontenc}
\usepackage[dvips]{graphicx}
\usepackage{amsmath}
\usepackage{textcomp}
\usepackage{amssymb}

% for multirow command
\usepackage{multirow}

% for arrays
\usepackage{array}

% for long tables
\usepackage{longtable}

% for booktabs
\usepackage{booktabs}

\title{Table Construction}
\author{Lothar Rubusch}
\date{2006-10-11}

\begin{document}
% write (almost) everything in sansserif font
\sffamily

% make title
\maketitle

% no indent at line start
\parindent0mm

\section{Plain}
\begin{tabular}{ccccccccc}
x & x & x & x & Col & x & x & x & x \\ 
Row & Row & Row & Row & X-ing & Row & Row & Row & Row \\ 
x & x & x & x & Col & x & x & x & x \\ 
x & x & x & x & Col & x & x & x & x \\ 
x & x & x & x & Col & x & x & x & x \\ 
\end{tabular}
\newline


\section{Cols}
\begin{tabular}{|c|c|c|c|c|c|c|c|c|}
x & x & x & x & Col & x & x & x & x \\ 
Row & Row & Row & Row & X-ing & Row & Row & Row & Row \\ 
x & x & x & x & Col & x & x & x & x \\ 
x & x & x & x & Col & x & x & x & x \\ 
x & x & x & x & Col & x & x & x & x \\ 
\end{tabular}
\newline


\section{Rows}
\begin{tabular}{ccccccccc} \hline
x & x & x & x & Col & x & x & x & x \\ \hline 
Row & Row & Row & Row & X-ing & Row & Row & Row & Row \\ \hline 
x & x & x & x & Col & x & x & x & x \\ \hline
x & x & x & x & Col & x & x & x & x \\ \hline
x & x & x & x & Col & x & x & x & x \\ \hline
\end{tabular}
\newline


\section{Grid}
\begin{tabular}{|c|c|c|c|c|c|c|c|c|} \hline
x & x & x & x & Col & x & x & x & x \\ \hline 
Row & Row & Row & Row & X-ing & Row & Row & Row & Row \\ \hline 
x & x & x & x & Col & x & x & x & x \\ \hline
x & x & x & x & Col & x & x & x & x \\ \hline
x & x & x & x & Col & x & x & x & x \\ \hline
\end{tabular}
\newline


\section{Header}
\begin{tabular}{|c||cccccccc|} \hline
x & x & x & x & Col & x & x & x & x \\ \hline \hline
Row & Row & Row & Row & X-ing & Row & Row & Row & Row \\
x & x & x & x & Col & x & x & x & x \\
x & x & x & x & Col & x & x & x & x \\
x & x & x & x & Col & x & x & x & x \\ \hline
\end{tabular}
\newline


\section{Merge Cols (1)}
\begin{tabular}{|c|c|c|c|c|c|c|c|c|} \hline
x & x & x & x & x & x & x & x & x \\ \hline
x & x & x & x & \multicolumn{2}{c}{Left border is static!} & x & x & x \\ \hline
x & x & x & x & x & x & x & x & x \\ \hline
x & x & x & x & \multicolumn{2}{|c}{Left doesn't matter!} & x & x & x \\ \hline
x & x & x & x & x & x & x & x & x \\ \hline
x & x & x & x & \multicolumn{2}{|c|}{Both} & x & x & x \\ \hline
x & x & x & x & x & x & x & x & x \\ \hline
\multicolumn{2}{c}{First left border misses!} & x & x & x & x & x & x & x \\ \hline
x & x & x & x & x & x & x & x & x \\ \hline
\multicolumn{2}{|c}{First left fixed!} & x & x & x & x & x & x & x \\ \hline
x & x & x & x & x & x & x & x & x \\ \hline
x & x & x & x & \multicolumn{2}{c|}{Right fixed, too!} & x & x & x \\ \hline
x & x & x & x & x & x & x & x & x \\ \hline
\end{tabular}
\newline


\section{Merge Cols (2)}
\begin{tabular}{|c|c|c|c|c|c|c|c|c|} \hline
x & x & x & x & Single Cell & x & x & x & x \\ \hline
x & x & x & x & \multicolumn{2}{c|}{Merged Col Cell} & x & x & x \\ \hline
x & x & x & x & \multicolumn{3}{c|}{Merged Col Cell} & x & x \\ \hline
x & x & x & x & \multicolumn{4}{c|}{Merged Col Cell} & x \\ \hline
x & x & x & x & \multicolumn{5}{c|}{Merged Col Cell} \\ \hline
x & x & x & x & x & x & x & x & x \\ \hline
x & x & x & x & \multicolumn{1}{c}{No border} & \multicolumn{1}{c}{No border} & \multicolumn{1}{c}{No border} & \multicolumn{1}{c}{No border} & \multicolumn{1}{c|}{No border} \\ \hline
x & x & x & x & \multicolumn{1}{c}{No border} & \multicolumn{1}{c}{No border} & \multicolumn{1}{c}{No border} & \multicolumn{1}{c|}{No border} & x \\ \hline
x & x & x & x & \multicolumn{1}{c}{No border} & \multicolumn{1}{c}{No border} & \multicolumn{1}{c|}{No border} & x & x \\ \hline
x & x & x & x & \multicolumn{1}{c}{No border} & No border & x & x & x \\ \hline
x & x & x & x & \multicolumn{1}{c|}{No border} & x & x & x & x \\ \hline
x & x & x & x & x & x & x & x & x \\ \hline
\end{tabular}
\newline


\section{Merge Rows}
\begin{tabular}{cccccccc|c|} \hline
\multicolumn{1}{|c|}{x} & x & x & x & x & x & x & x & x \\ \cline{1-3}
x & x & x & x & x & x & x & x & x \\ 
x & x & x & x & x & x & x & x & x \\ \cline{1-2} \cline{5-6}
x & x & x & x & x & x & x & x & x \\ 
x & x & x & x & x & x & x & x & x \\ \cline{1-1}
\multicolumn{1}{|c|}{x} & x & x & x & x & x & x & x & x \\ \cline{1-2}
\multicolumn{1}{|c|}{x} & \multicolumn{1}{c|}{x} & x & x & x & x & x & x & x \\ \cline{1-3}
\multicolumn{1}{|c|}{x} & \multicolumn{1}{c|}{x} & \multicolumn{1}{c|}{x} & x & x & x & x & x & x \\ \hline
\multicolumn{1}{|c|}{x} & \multicolumn{1}{c|}{x} & \multicolumn{1}{c|}{x} & x & x & x & x & x & x \\ \cline{4 - 9}
\multicolumn{1}{|c|}{x} & \multicolumn{1}{c|}{x} & \multicolumn{1}{c|}{x} & \multicolumn{1}{|c|}{x} & x & x & x & x & x \\ \hline
\end{tabular}
\newline


\section{Merge Cells: Multirow}
Uses package multirow!
\newline

\begin{tabular}{|c|c|c|c|c|} \hline
x & x & \multirow{3}{14mm}{Beware of hline} & x & x \\ \hline
x & x &  & x & x \\ \hline 
x & x &  & x & x \\ \hline
x & x & x & x & x \\ \hline
x & x & x & \multirow{2}{14mm}{Better!} & x \\ \cline{1-3} \cline{5-5}
x & x & x & & x \\ \hline
\end{tabular}
\newline


\section{Center}
\begin{center}
\begin{tabular}{ccccccccc}
x & x & x & x & Col & x & x & x & x \\ 
Row & Row & Row & Row & X-ing & Row & Row & Row & Row \\ 
x & x & x & x & Col & x & x & x & x \\ 
x & x & x & x & Col & x & x & x & x \\ 
x & x & x & x & Col & x & x & x & x \\ 
\end{tabular}
\newline
\end{center}


\section{Left-Center-Right}
\begin{center}
\begin{tabular}{|c|l|c|r|c|c|} \hline
loooooong word & loooooong word & loooooong word & loooooong word & word & word \\ \hline \hline
center & l & c & r & center & center \\ \hline
center & l & c & r & center & center \\ \hline
center & l & c & r & center & center \\ \hline
\end{tabular}
\newline
\end{center}


\section{Static size}
\begin{center}
\begin{tabular}{|c|c|p{7.5 cm}|c|c|} \hline
x & x & x & x & x \\ \hline
x & x & x & x & x \\ \hline 
x & x & x & x & x \\ \hline
x & x & x & x & x \\ \hline
x & x & x & x & x \\ \hline
x & x & x & x & x \\ \hline
\end{tabular}
\newline
\end{center}


\section{Static Size and Ending Sequence}
\begin{center}
\begin{tabular}{|c|c|p{7.5 cm}@{end of cell}|c|c|} \hline
x & x & x & x & x \\ \hline
x & x & x & x & x \\ \hline 
x & x & x & x & x \\ \hline
x & x & x & x & x \\ \hline
x & x & x & x & x \\ \hline
x & x & x & x & x \\ \hline
\end{tabular}
\newline
\end{center}


\pagebreak
\section{Position}
Possible positions are c,t or b.
\begin{quote}
Call me Ishmael. Some years ago- never mind how long precisely- having little or no money in my purse, and nothing particular to interest me on shore, I thought I would sail about a little and see the watery part of the world. 
\newline
\end{quote}

Call me Ishmael. Some years ago- never mind how long precisely- having 
\begin{tabular}[c]{|c|}
little \\
no \\
a lot of \\
\end{tabular}
money in my purse, and nothing particular to interest me on shore, I thought I would sail about a little and see the watery part of the world.
\newline

Call me Ishmael. Some years ago- never mind how long precisely- having 
\begin{tabular}[t]{|c|}
little \\
no \\
a lot of \\
\end{tabular}
money in my purse, and nothing particular to interest me on shore, I thought I would sail about a little and see the watery part of the world. 
\newline

Call me Ishmael. Some years ago- never mind how long precisely- having 
\begin{tabular}[b]{|c|}
little \\
no \\
a lot of \\
\end{tabular}
money in my purse, and nothing particular to interest me on shore, I thought I would sail about a little and see the watery part of the world.
\newline 


\section{Repeted Declaration}
The custom way..
\begin{center}
\begin{tabular}{|r@{,}l|r@{,}l|r@{,}l|r@{,}l|} \hline
1&2 & 3&4 & 5&6 & 7&8 \\ \hline
9&10 & 11&12 & 13&14 & 15&16 \\ \hline
\end{tabular}
\end{center}

The new way...
\begin{center}
\begin{tabular}{|*{4}{r@{,}l|}} \hline
1&2 & 3&4 & 5&6 & 7&8 \\ \hline
9&10 & 11&12 & 13&14 & 15&16 \\ \hline
\end{tabular}
\end{center}

Serial way using a '.' instead of a ','
\begin{center}
\begin{tabular}{|*{4}{r@{.}l|}} \hline
1&2 & 3&4 & 5&6 & 7&8 \\ \hline
9&10 & 11&12 & 13&14 & 15&16 \\ \hline
\end{tabular}
\end{center}


\section{Tabbing}
\begin{tabbing}
% define tabs after 'Command' and 'Explanation'
Command \quad\= 
Explanation \qquad\qquad\= 
\kill
% write content
\textbf{Command} \> \textbf{Explanation}  \> \textbf{Use} \\
$\backslash$= \> set a tab \> word $\backslash$= \\
$\backslash$quad$\backslash$=  \> define a tab \> word $\backslash$quad$\backslash$= \\ \newline
$\backslash$> \> next tab field \> $\backslash$> next word \\
$\backslash$< \> previous tab field \> $\backslash$< prev word \\
$\backslash$+ \> left side one \> end of row $\backslash$+ \+ \\
position to the left \-\\
$\backslash$- \> right side one \> end of row $\backslash$- \+\\
position to the right \- \\
$\backslash$kill \> spacer \> after tab declaration\\
\end{tabbing}

The problem of tabbing is:
\begin{tabbing}
This \= text is ok \\
This text is \> screwed up!!
\newline
\end{tabbing}


\section{Long Tables}
Uses package longtable.! \\
\begin{longtable}{|l|cc|} \hline
end & first & head\\ \hline
\endfirsthead

all & pages & head\\ \hline
\endhead

all & pages & footer\\ \hline
\endfoot

end & last & foot\\ \hline
\endlastfoot

% content
x & x & x\\ 
x & x & x\\ 
x & x & x\\ 
x & x & x\\ 
x & x & x\\ 
x & x & x\\ 
x & x & x\\ 
x & x & x\\ 
x & x & x\\ 
x & x & x\\ 
x & x & x\\ 
x & x & x\\ 
x & x & x\\ 
x & x & x\\ 
x & x & x\\ 
x & x & x\\ 
x & x & x\\ 
x & x & x\\ 
x & x & x\\ 
x & x & x\\ 
x & x & x\\ 
x & x & x\\ 
x & x & x\\ 
x & x & x\\ 
x & x & x\\ 
x & x & x\\ 
x & x & x\\ 
x & x & x\\ 
x & x & x\\ 
x & x & x\\ 
x & x & x\\ 
x & x & x\\ 
x & x & x\\ 
x & x & x\\ 
x & x & x\\ 
x & x & x\\ 
x & x & x\\ 
x & x & x\\ 
x & x & x\\ 
x & x & x\\ 
x & x & x\\ 
x & x & x\\ 
x & x & x\\ 
x & x & x\\ 
x & x & x\\ 
x & x & x\\ 
x & x & x\\ 
x & x & x\\ 
x & x & x\\ 
x & x & x\\ 
x & x & x\\ 
x & x & x\\ 
x & x & x\\ 
x & x & x\\ 
x & x & x\\ 
x & x & x\\ 
x & x & x\\ 
x & x & x\\ 
x & x & x\\ 
x & x & x\\ 
x & x & x\\ 
x & x & x\\ 
x & x & x\\ 
x & x & x\\ 
x & x & x\\ 
x & x & x\\ 
x & x & x\\ 
x & x & x\\ 
x & x & x\\ 
x & x & x\\ 
x & x & x\\ 
x & x & x\\ 
x & x & x\\ 
x & x & x\\ 
x & x & x\\ 
x & x & x\\ 
x & x & x\\ 
x & x & x\\ 
x & x & x\\ 
x & x & x\\ 
x & x & x\\ 
x & x & x\\ 
x & x & x\\ 
x & x & x\\ 
x & x & x\\ 
x & x & x\\ 
x & x & x\\ 
x & x & x\\ 
x & x & x\\ 
x & x & x\\ 
x & x & x\\ 
x & x & x\\ 
x & x & x\\ 
x & x & x\\ 
x & x & x\\ 
x & x & x\\ 
x & x & x\\ 
\end{longtable}


\section{Array}
\setlength{\extrarowheight}{4pt}
\begin{tabular}{|p{2cm}|m{2cm}|b{2cm}|} \hline
p\{\dots\} & m\{\dots\} & b\{\dots\} \\ \hline
grounded & centered & leading \\ \hline
Call me Ishmael. & Call me Ishmael. & Call me Ishmael. \\ \hline
Call me Ishmael. & Call me Ishmael. & Call me Ishmael. \\ \hline
Call me Ishmael. & Call me Ishmael. & Call me Ishmael. \\ \hline
\end{tabular}
\newline


\section{Array: Format}
Format an array using comments or commands at the column declaration. 
\begin{itemize}
 \item >\{comment\} sets it before a column,
 \item <\{comment\} sets it after a column (never used it)
 \item !\{comment\} between the columns (never ever used that one)
\end{itemize}

\setlength{\extrarowheight}{4pt}
\begin{tabular}{|>{\bfseries BEFORE}p{2cm}|m{2cm}|>{\large\bfseries}b{2cm}|} \hline
p\{\dots\} & m\{\dots\} & b\{\dots\} \\ \hline
grounded & centered & leading \\ \hline
Call me Ishmael. & Call me Ishmael. & Call me Ishmael. \\ \hline
Call me Ishmael. & Call me Ishmael. & Call me Ishmael. \\ \hline
Call me Ishmael. & Call me Ishmael. & Call me Ishmael. \\ \hline
\end{tabular}
\newline


\pagebreak
\section{Scientific Booktabs}
Uses package booktabs! Tables for scientific articles.
\begin{itemize}
 \item $\backslash$toprule a line on top of the table
 \item $\backslash$midrule a line within the table
 \item $\backslash$bottomrule a line at the bottom of the table
 \item $\backslash$cmimidrule[trim]\{ col.start - col.end \} works like $\backslash$cline, trim defines where to place the gap (l,r or lr possible)
 \item $\backslash$addlinespace generates an additional linespace
\end{itemize}

\begin{tabular}{cclr}
\toprule

\multicolumn{3}{c}{bla} \\

% use a doubled cmiruler with gap
\cmidrule(r){1-3}\morecmidrules\cmidrule(r){1-3}

bla & bla & bla & weight in g \\

\midrule
item & description & spec. & 1.1 \\
item & description & spec. & 2.2 \\
item & description & spec. & 3.3 \\
item & description & spec. & 4.4 \\

\addlinespace

\bottomrule
\end{tabular}



\end{document}
